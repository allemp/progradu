\documentclass[utf8,english]{gradu3}
% If you are writing a Bachelor's Thesis, use the following instead:
%\documentclass[utf8,bachelor,english]{gradu3}

\usepackage{graphicx} % for including pictures

\usepackage{amsmath} % useful for math (optional)

\usepackage{booktabs} % good for beautiful tables

% NOTE: This must be the last \usepackage in the whole document!
\usepackage[bookmarksopen,bookmarksnumbered,linktocpage]{hyperref}

\addbibresource{bibliography.bib} % The file name of your bibliography database

\begin{document}

\title{Automatic speech recognition in physics teacher education}
\translatedtitle{Automaattinen puheentunnistus fysiikan opettajakoulutuksessa}
\studyline{Educational Technology}
\avainsanat{%
  TODO: Avainsanat}
\keywords{TODO: Keywords}
\tiivistelma{%
  TODO: Tiivistelmä
}
\abstract{%
  TODO: Abstract
}

\author{Aleksander Lempinen}
\contactinformation{\texttt{aleksander.lempinen@outlook.com}}
% use a separate \author command for each author, if there is more than one
\supervisor{Tommi Kärkkäinen}
\supervisor{Daniela Caballero}
\supervisor{Jouni Viiri}


% use a separate \supervisor command for each supervisor, if there
% is more than one

\maketitle

\begin{thetermlist}
\item[TODO] TODO: Glossary
\end{thetermlist}

\mainmatter

\chapter{Introduction}
Modern physics classroom instruction attempts to improve learning by reducing the cognitive load, which is done by providing a clear organizational structure for the factual knowledge and linking new material to previously known ideas \parencite{wieman2005transforming}. The goal of physics instruction is to help students become experts capable of solving problems \parencite{fischer2014quality,wieman2005transforming}. While lecture based instruction is often not very effective for retaining new knowledge \parencite{wieman2005transforming}, the content structure and relationships between concepts presented by the teacher positively correlate with student learning gains when analysed with conceptual network analysis \parencite{fischer2014quality}. 

For a long time automatic speech recognition (ASR) was outperformed by human speech recognition (HSR), but with the recent developments in the deep learning approach the error rate of automatic speech recognition and human speech recognition is almost the same in certain tasks \parencite{spille2018comparing}. This however is language specific. Work on automatic speech recognition with conversational Finnish started in 2012 and word error rate of 27.1\% was achieved by 2017 \parencite{enarvi2018modeling}. 

In natural language processing the creation of treebanks such as Turku Dependency Treebank and FinnTreeBank in the past decade have allowed for development of natural language processing toolkits with adequate performance with Finnish language such as FinnPos \parencite{silfverberg2016finnpos} and TurkuNLP \parencite{kanerva2018turku}. Lemmatization in the new toolkits is of particular interest, because it is essential for real world tasks in inflective languages such as Finnish \parencite{kanerva2018turku}.

Manually transcribing and preprocessing lessons from audio data for analysis is a very laborious task. The aim of this study is to develop a pipeline for studying physics lesson content structures and relationships between physics concepts with conceptual network analysis using automatic speech recognition and natural language processing. This is done to provide a new tool for analysing physics instruction for research and teacher education purposes. 

\chapter{Physics instruction quality}
TODO: Paragraph about science education and physics education history
TODO: Paragraph about the change of science and physics education and modern physics education
TODO: Paragraph about educational material and pedagogical tools
TODO: Paragraph about physiscs instruction

TODO: Paragraph about cognitive structure and content structure
TODO: Paragraph about how cognitive structure fits with education (geeslin, shavelson)
TODO: Paragraph about linking cognitive structure of students and teachers
TODO: Paragraph about content structure with written text with content structure (geeslin, shavelson, wieman)

TODO: Paragraph about pedagogical link making

\section{Pedagogical link making}
TODO: \parencite{scott2011pedagogical}

\section{Conceptual network analysis}
TODO: \parencite{mclinden2013concept,fischer2014quality, vargas}

\chapter{Speech as data}
TODO: AI overview, AI history, ASR and NLP concepts

\section{Automatic speech recognition}
Automatic speech recognition (ASR), sometimes called speech to text, is a classification task, where the goal is to predict what was said from the audio signal of speech. Early ASR systems had an acoustic model which detected different sounds also known as phonemes to recognize numbers, some vowels and consonants for a single speaker \parencite{juang2005automatic}. The later addition of a language model based on statistical grammar or syntax helped to predict the correct word based on what words previously appeared in the sentence \parencite{juang2005automatic}. Modern ASR systems utilize the fact that sentences are sequences of words and words are sequences of phonemes \parencite{bengio2014word}. This is a difficult machine learning task because of a large search space, large vocabulary, undetermined length of word sequences and problems related to aligning speech signal to the text \parencite{enarvi2018modeling}.

Speech is highly variable even with a single speaker due to noise, but different pronounciations and accents mean that the audio signal will be different despite the same words being spoken \parencite{juang2005automatic}. 

\subsection{Finnish speech data}
TODO: Conversational vs official Finnish

\section{Natural language processing}
TODO: NLP overview \parencite{silfverberg2016finnpos, kanerva2018turku}

\subsection{Word embedding}
TODO: word2vec

\subsection{Lemmatization and stemming}
TODO: Snowball vs TurkuNLP lemmatization

\subsection{Finnish NLP data}
TODO: Data from different sources, treebanks for parsing vs. word lists/text

\chapter{Methods}

%https://www.kdnuggets.com/2019/08/neighbours-machine-learning-graphs.html
\section{Data}
  TODO: \parencite{fischer2014quality}
\section{Automatic speech recognition}
  TODO: \parencite{enarvi2018modeling}
\section{Lemmatization and stemming}
  TODO: TurkuNLP \parencite{kanerva2018turku}
  TODO: libvoikko https://github.com/voikko/corevoikko/tree/master/libvoikko
  TODO: FinnPOS \parencite{silfverberg2016finnpos}
\section{Word frequency}
\section{Network analysis}
Social network analysis has been used in physics education research for example to study student collaboration \parencite{vargas2018correlation}, interactions within student communities \parencite{brewe2012investigating} 

\chapter{Results}

TODO:
\chapter{Conclusions}
TODO:

Research is being done in collaboration with the Department of Teacher Education at University of Jyväskylä and Centro de Investigación Avanzada en Educación (CIAE) at University of Chile.
 
\printbibliography

\end{document}
