\documentclass[utf8,english]{gradu3}
% If you are writing a Bachelor's Thesis, use the following instead:
%\documentclass[utf8,bachelor,english]{gradu3}

\usepackage{graphicx} % for including pictures

\usepackage{amsmath} % useful for math (optional)

\usepackage{booktabs} % good for beautiful tables

% NOTE: This must be the last \usepackage in the whole document!
\usepackage[bookmarksopen,bookmarksnumbered,linktocpage]{hyperref}

\addbibresource{bibliography.bib} % The file name of your bibliography database

\begin{document}

\title{Automatic speech recognition in physics teacher education}
\translatedtitle{Automaattinen puheentunnistus fysiikan opettajakoulutuksessa}
\studyline{Educational Technology}
\avainsanat{%
  TODO: Avainsanat}
\keywords{TODO: Keywords}
\tiivistelma{%
  TODO: Tiivistelmä
}
\abstract{%
  TODO: Abstract
}

\author{Aleksander Lempinen}
\contactinformation{\texttt{aleksander.lempinen@outlook.com}}
% use a separate \author command for each author, if there is more than one
\supervisor{Tommi Kärkkäinen}
\supervisor{Daniela Caballero}
\supervisor{Jouni Viiri}


% use a separate \supervisor command for each supervisor, if there
% is more than one

\maketitle

\begin{thetermlist}
\item[TODO] TODO: Glossary
\end{thetermlist}

\mainmatter

\chapter{Introduction}
Modern physics classroom instruction attempts to improve learning by reducing the cognitive load, which is done by providing a clear organizational structure for the factual knowledge and linking new material to previously known ideas \parencite{wieman2005transforming}. The goal of physics instruction is to help students become experts capable of solving problems \parencite{fischer2014quality,wieman2005transforming}. While lecture based instruction is often not very effective for retaining new knowledge \parencite{wieman2005transforming}, the content structure and relationships between concepts presented by the teacher positively correlate with student learning gains when analysed with conceptual network analysis \parencite{fischer2014quality}. 

For a long time automatic speech recognition (ASR) was outperformed by human speech recognition (HSR), but with the recent developments in the deep learning approach the error rate of automatic speech recognition and human speech recognition is almost the same in certain tasks \parencite{spille2018comparing}. This however is language specific. Work on automatic speech recognition with conversational Finnish started in 2012 and word error rate of 27.1\% was achieved by 2017 \parencite{enarvi2018modeling}. 

In natural language processing the creation of treebanks such as Turku Dependency Treebank and FinnTreeBank in the past decade have allowed for development of natural language processing toolkits with adequate performance with Finnish language such as FinnPos \parencite{silfverberg2016finnpos} and TurkuNLP \parencite{kanerva2018turku}. Lemmatization in the new toolkits is of particular interest, because it is essential for real world tasks in inflective languages such as Finnish \parencite{kanerva2018turku}.

Manually transcribing and preprocessing lessons from audio data for analysis is a very laborious task. The aim of this study is to develop a pipeline for studying physics lesson content structures and relationships between physics concepts with conceptual network analysis using automatic speech recognition and natural language processing. This is done to provide a new tool for analysing physics instruction for research and teacher education purposes. 

\chapter{Physics instruction quality}

\section{Pedagogical link making}

\section{Conceptual network analysis}

\chapter{Speech as data}

\section{Automatic speech recognition}
Automatic speech recognition (ASR), sometimes called speech to text, is a classification task, where the goal is to predict what was said from the audio signal of speech. Early ASR systems had an acoustic model which detected different sounds also known as phonemes to recognize numbers, some vowels and consonants for a single speaker \parencite{juang2005automatic}. Improvements in the acoustic model allowed for introduction of speaker-independent ASR \parencite{benzeghiba2007automatic,juang2005automatic}. The later addition of a language model based on statistical grammar and syntax helped more accurately predict the correct word based on what words previously appeared in the sentence \parencite{juang2005automatic}. Modern ASR systems utilize the fact that sentences are sequences of words and words are sequences of phonemes \parencite{bengio2014word}. 

Sequence based models such as hidden markov models (HMM) are most commonly used, but deep learning approaches using recurrent neural networks (RNN) and long short-term memory (LSTM) networks gaining popularity for acoustic models, language models and end to end text-to-speech models \parencite{bengio2014word,enarvi2018modeling}. Deep learning approach is more data-driven and relies on fewer assumptions, but instead requires more data for training \parencite{bengio2014word}. This might be impractical if training data is limited. Depending on the architecture, an ASR system might be capable of either transcription, keyword spotting or both \parencite{juang2005automatic,enarvi2018modeling}.

ASR is a difficult machine learning task because of a large search space, large vocabulary, undetermined length of word sequences and problems related to aligning speech signal to the text \parencite{enarvi2018modeling}. Speech is highly variable even with a single speaker due to noise, but different pronounciations and accents mean that the audio signal will be different despite the same words being spoken \parencite{juang2005automatic}. Accents, dialects, emotional state, gender and casual speech slurring in spontaneous speech bring a lot of variation which makes conversational speech especially difficult compared to standard pronounciations and vocabulary \parencite{benzeghiba2007automatic, juang2005automatic}. Speaker-dependent systems are typically more accurate than speaker-independent systems \parencite{benzeghiba2007automatic,enarvi2018modeling}.

\section{Natural language processing}
TODO: NLP overview \parencite{silfverberg2016finnpos, kanerva2018turku}


\section{Finnish speech}
TODO: Conversational vs official Finnish
Finnish language is particularly difficult for ASR because words are formed by concatenating smaller \parencite{enarvi}



\chapter{Methods}

%https://www.kdnuggets.com/2019/08/neighbours-machine-learning-graphs.html
\section{Data}
The data set consists of 25 Finnish physics lessons on the topic of "Relation between electrical energy and power" from Quality of Instruction in Physics (QuIP) project \parencite{helaakoski20146}. In addition to the teacher speech recording, test results from each student are available from before and after the lesson. 

\section{Preprocessing}
\subsection{Network analysis}
\subsection{Adjacency matrix}
\section{Supervised machine learning}
\section{Clustering}
\section{Validation}


\chapter{Results}

TODO:
\chapter{Conclusions}
TODO:

Research is being done in collaboration with the Department of Teacher Education at University of Jyväskylä and Centro de Investigación Avanzada en Educación (CIAE) at University of Chile.
 
\printbibliography

\end{document}
