\documentclass[utf8,english]{gradu3}
% If you are writing a Bachelor's Thesis, use the following instead:
%\documentclass[utf8,bachelor,english]{gradu3}

\usepackage{graphicx} % for including pictures

\usepackage{amsmath} % useful for math (optional)

\usepackage{booktabs} % good for beautiful tables

% NOTE: This must be the last \usepackage in the whole document!
\usepackage[bookmarksopen,bookmarksnumbered,linktocpage]{hyperref}

\addbibresource{bibliography.bib} % The file name of your bibliography database

\begin{document}

\title{Automatic speech recognition in physics teacher education}
\translatedtitle{Automaattinen puheentunnistus fysiikan opettajakoulutuksessa}
\studyline{Educational Technology}
\avainsanat{%
  TODO: Avainsanat}
\keywords{TODO: Keywords}
\tiivistelma{%
  TODO: Tiivistelmä
}
\abstract{%
  TODO: Abstract
}

\author{Aleksander Lempinen}
\contactinformation{\texttt{aleksander.lempinen@outlook.com}}
% use a separate \author command for each author, if there is more than one
\supervisor{Tommi Kärkkäinen}
\supervisor{Daniela Caballero}
\supervisor{Jouni Viiri}


% use a separate \supervisor command for each supervisor, if there
% is more than one

\maketitle

\begin{thetermlist}
\item[TODO] TODO: Glossary
\end{thetermlist}

\mainmatter

\chapter{Introduction}

abrakadabra

\chapter{Literature review}

The method for literature review is to start from a small set of relevant literature and continue the search for previous literature using their references and newer literature from "cited by" list provided by Google Scholar.

TODO: Automatic speech recognition \parencite{hirsimaki2009importance,Mansikkaniemi2017}

TODO: Natural language processing \parencite{gambhir2017recent,tuhkala2018semi}

Social network analysis is a popular network and graph theory based technique used in social sciences \parencite{borgatti2009network} and can be used to study relationships between people or other entities. It has been applied in physics education to study the effects of collaboration between students and their academic performance \parencite{vargas2018correlation}. Techniques from social network analysis have also been applied to study relationships between other entities than people such as concepts or topics \parencite{mclinden2013concept}. A proof of concept has already been done using the same QuIP data and Aalto ASR using social network analysis to visualize the relationships between physics keywords \parencite{caballero2017asr}.



\chapter{Research topic}

TODO: ?

\chapter{Research questions}
The aim of the research is to develop a method for automatic feedback of lesson quality from automatic speech recognition data. Due to the nature of automatically generated transcripts, feature engineering and feature extraction is a critical preprocessing step which will affect everything down the data analysis pipeline.

This research has the following research questions:


\begin{enumerate}
  \item What preprocessing steps can improve the results of network analysis?
  \item Can unsupervised machine learning such as association rule mining provide additional insight?
\end{enumerate}

\chapter{Research strategy}
The research strategy is learning analytics/data analytics.

\chapter{Dataset}

The dataset is from QuIP project \parencite{fischer2014quality} consisting of 25 Finnish lesson transcripts generated by Aalto ASR \parencite{hirsimaki2009importance,Mansikkaniemi2017} and data from 756 students. The topic of the lesson is related to electricity and students were tested before and after the lesson with a standard tests. Transcript output from Aalto ASR has 5 second splits and there is a set of physics related keywords available.


\chapter{Data analysis}

The bulk of data analysis will be to expand existing network analysis on QuIP data with different kinds of preprocessing to obtain different features and relationships for the graph measures. In addition the results will be compared to unsupervised machine learning methods such as association rule mining.

\chapter{Results}

The expected results of this research would be improved preprocessing approaches for network analysis of physics lesson transcripts after automatic speech recognition and alternative machine learning based approaches to network analysis.

\chapter{Conclusions}

Research is being done in collaboration with the Department of Teacher Education at University of Jyväskylä and Centro de Investigación Avanzada en Educación (CIAE) at Univeristy of Chile.
 
\printbibliography

\end{document}
